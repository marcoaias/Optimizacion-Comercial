\documentclass[12pt]{article}


\usepackage{booktabs, multirow} % for borders and merged ranges
\usepackage{soul}% for underlines
\usepackage[table]{xcolor} % for cell colors
\usepackage{changepage,threeparttable}


\usepackage[official]{eurosym}


\usepackage{color}
% \usepackage[usenames,dvipsnames]{xcolor}


\newcommand{\hlc}[2][yellow]{ {\sethlcolor{#1} \hl{#2}} }

\usepackage{commath}

\usepackage[utf8]{inputenc}
\newtheorem{regla}{Regla}

\usepackage[spanish,es-tabla]{babel}

\usepackage{amsmath, amssymb, amsfonts}

\usepackage{mathtools}
\DeclarePairedDelimiter{\ceil}{\lceil}{\rceil}

\usepackage{tikz, pgfplots}
\pgfplotsset{compat=1.5}


\usepackage{hyperref}


\definecolor{ForestGreen}{RGB}{0,155,85}
\definecolor{RoyalBlue}{RGB}{0,113,188}



% \newcommand{\hlc}[2][yellow]{ {\sethlcolor{#1} \hl{#2}} }


\title{Optimización Comercial}
\author{Marco Aias et al.}
\date{\today}



\begin{document}
\maketitle



\section{A realizar}

Sea una cierta compañía de trenes de alta velocidad; la cual ofrece servicios de transporte entre distintas ciudades españolas, desde y hacia Madrid.\\

Sea el trayecto que nos compete Madrid-Bilbao, hemos de determinar la mejor manera de comercializar los asientos de tal tren según los distintos productos ofrecidos.\\

Consiste en determinar la cantidad de asientos esperados a ser vendidos para cada producto, con tal de que el total vendido sea óptimo. \\

Para tal se empleará un modelo de manejo de ingresos; ámpliamente usado, el algoritmo heurístico EMSR-b (Expected Marginal Seat Revenue).\\

Con tal información de demanda esperada para cada producto, se ha de determinar el número de vagones a configurar para el tren, atendiendo a las tasas implicadas.


\section{Productos}



\begin{adjustwidth}{-2.5 cm}{-2.5 cm}\centering
\begin{threeparttable}[!htb]

\small
\begin{tabular}{lccccc}\toprule
Clase de reserva &Tarifa [\euro] &Cambios permitidos &Sala Vip &Fast Track &Elección asiento \\\midrule
A &180 &Sí &Sí &Sí &Sí \\
B &130 &Solo 1 (penalización 25 \euro) &No &Sí &Sí \\
C &100 &Solo 1 (penalización 60 \euro) &No &No &Sí \\
D &80 &No &No &No &No \\
E &40 &No &No &No &No \\
\bottomrule
\end{tabular}
\caption{Servicios y precio de cada clase.}\label{tab:clases}
\end{threeparttable}

\end{adjustwidth}


\section{Demanda}

Un departamento de \emph{forecasting} ha proyectado para un día en concreto la demanda esperada para cada producto. Ésta viene recogida en la tabla \ref{tab:demanda}.\\

Se asume que la demanda para cada clase es independiente del resto; se asume la llegada de los clientes según la tarifa: primero se venden los asientos de la clase E, antes de ser vendidos aquellos con una tarifa más elevada.



\begin{table}[!htp]\centering

%\small
\begin{tabular}{lrrr}\toprule
Clase de reserva &$\mu$ &$\sigma$ \\\midrule
A &\hl{ ? } &\hl{ ? } \\
B &87 &8 \\
C &89 &9 \\
D &\hl{ ? } &\hl{ ? } \\
E &60 &9 \\
\bottomrule
\end{tabular}

\caption{Proyección de demanda para cada clase. (Trayecto MAD - BIO)}\label{tab:demanda}
\end{table}





\begin{align}
\sigma_1 &= e^{\mu / 7} + 2\label{eq:expbio}\\[8px]
\sigma_2 &= \dfrac{1}{10}\mu^{3} -20\mu + 20\label{eq:cubbio}
\end{align}

\vspace{10pt}

La tabla \ref{tab:demanda} y las ecuaciones \eqref{eq:expbio} y \eqref{eq:cubbio} corresponden al trayecto MAD - BIO; las ecuaciones aplican a las clases A y D.



\begin{figure}
\centering
\begin{tikzpicture}
\begin{axis} [
title=Curvas de demanda,
xlabel={$\mu$},
ylabel={$\sigma$},
ymax=20,
ymin=5,
xmax=20,
xmin=5,
%domain=5:35,
samples=200,
width=12cm, height=4cm,
legend pos=north west,
]






\addplot [red, thick,domain=5:35, restrict y to domain=1:30] {exp(x/7)+2};
\addlegendentry{$e^{\mu / 7} + 2$}

\addplot [blue, thick,domain=5:35, restrict y to domain=-90:120] {0.1*x^3 - 20*x + 20};
\addlegendentry{$0.1\mu^{3} -20\mu + 20$}



\end{axis}


\end{tikzpicture}
\caption{Representación de las ecuaciones \eqref{eq:expbio} y \eqref{eq:cubbio}}\label{fig:curvas}
\end{figure}


Sea el espacio de la solución:
$$\left\{5 \leq \mu \leq 35\right\}, \left\{1 \leq \sigma \leq 30\right\}$$



\begin{figure}
\centering
\begin{tikzpicture}
\begin{axis} [
title=Solución,
axis lines = middle,
xlabel={$\mu$},
ylabel={$\sigma_1 - \sigma_2$},
ymax=100,
ymin=-100,
%xmax=20,
%xmin=5,
domain=3:20,
samples=200,
width=12cm, height=4cm,
]

\addplot [cyan, thick] {exp(x/7)+2 - 0.1*x^3 + 20*x - 20};


\end{axis}


\end{tikzpicture}
\caption{Ecuación \eqref{eq:expbio} - \eqref{eq:cubbio}}\label{fig:sol}
\end{figure}




Según las figuras \ref{fig:curvas} y \ref{fig:sol}, la solución se encuentra en un entorno alrededor de $\mu \approx 14$.









\section{Probabilidad EMSR}


Para el cálculo de los EMSR(s), se requiere conocer la probabilidad de demanda; para lo cual el modelo toma los datos de demanda proyectados en la tabla \ref{tab:demanda} y forma una distribución normal de probabilidades.\\


La distribución gaussiana:

\begin{align}
f(x) &= \dfrac{1}{\sigma\sqrt{2\pi}}\exp\left({-\,\dfrac{1}{2}\left(\dfrac{x-\mu}{\sigma}\right)^2}\right) \label{eq:fgauss} \\[8pt]
P(t) &= \int_{-\infty}^{t} f(x) \dif x \underset{t\to +\infty}{=} 1 \label{eq:intgauss}
\end{align}


La integral \eqref{eq:intgauss} no se puede resolver de forma analítica, sino por aproximación por el \emph{método del trapecio} y también por el método de Simpson.\\



\begin{figure}
\centering
\begin{tikzpicture}
\begin{axis} [
title = Campana de Gauss,
axis lines = middle,
xlabel={$x$},
ylabel={$f(x)$},
ymax=0.5,
%ymin=5,
xmax=3,
xmin=-3,
%domain=3:20,
samples=200,
width=12cm,height=4cm,
]

\addplot [magenta, thick] {exp(-0.5*x^2)/(2*3.141592)^0.5};


\end{axis}


\end{tikzpicture}
\caption{Ecuación $\underset{\mu = 0,\,\sigma = 1}{\text{\eqref{eq:fgauss}}}$}\label{fig:campana}
\end{figure}



Dado que la función \eqref{eq:fgauss} es simétrica, la integral $\underset{t = \mu}{\text{\eqref{eq:intgauss}}}$ es igual a $\frac{1}{2}$. Entonces no es necesario calcular entre $(-\infty, t)$, sino $\frac{1}{2} + \int_\mu^t f(x)\dif x$ lo que alivia muchos recursos computacionales.

\begin{align}
P(t) = \dfrac{1}{2} + \int_\mu^t \underset{(\mu, \, \sigma)}{f(x)}\dif x \label{eq:probgauss}
\end{align}


\begin{figure}
\centering
\begin{tikzpicture}
\begin{axis} [
title=Probabilidad acumulada,
axis lines = middle,
legend style={nodes={scale=0.5, transform shape}},
xlabel={$t$},
ylabel={$\underset{(\mu,\, \sigma)}{P(t)}$},
ymax=1.1,
%ymin=5,
xmax=3,
xmin=-3,
%domain=3:20,
%samples=200,
width=12cm,height=4cm,
legend pos=south east,
]

\addplot [cyan, thick] table {data/data1.dat};
\addlegendentry{$(0,1)$}
\addplot [magenta, thick] table {data/data2.dat};
\addlegendentry{$(0,0.5)$}
\addplot [blue, thick] table {data/data3.dat};
\addlegendentry{$(1,1)$}
\addplot [red, thick] table {data/data4.dat};
\addlegendentry{$(-2,0.3)$}
\end{axis}


\end{tikzpicture}
\caption{Probabilidad acumulada en función de $t$ y $(\mu , \, \sigma)$ \eqref{eq:probgauss}}\label{fig:cdf}
\end{figure}



\section{Algoritmo EMSR-b}

Consiste en proteger suficientes \emph{asientos} de respectivas clases con tal de maximizar el ingreso total.\\

Se basa en la \emph{Regla de Littlewood}:

\begin{regla}
La razón entre las tarifas de respectivas clases ha de ser igual a la probabilidad de ser ocupado el asiento de la clase de la tarifa superior
\end{regla}


Ha de encontrarse el número de asientos protegidos ($t$) que satisfaga la probabilidad ($P(t)$) \eqref{eq:probgauss} de ser ocupados requerida por la razón de las respectivas tarifas.\\

Ampliamos la tabla \ref{tab:demanda} para calcular los parámetros relativos a los EMSR(s):\\





\begin{adjustwidth}{-2.5 cm}{-2.5 cm}\centering
\begin{threeparttable}[!htb]

%\small
\begin{tabular}{lrcccccr}\toprule
Clase$_i$ & Tarifa$_i$ & $\mu_i$ &$\sigma_i$ & $\text{tmp}_i$ & $\overline{\mu}_i$ & $\overline{\sigma}_i$ & Protección\\\midrule
A$_1$ & 180 & \hl{ ? } &\hl{ ? } & & & & \\
B$_2$ & 130 & 87 &8 & & & & \\
C$_3$ & 100 & 89 &9 & & & & \\
D$_4$ & 80 & \hl{ ? } &\hl{ ? } & & & & \\
E$_5$ & 40 & 60 &9 & - & - & - & -\\
\bottomrule
\end{tabular}

\caption{Valores EMSR-b}\label{tab:emsr}
\end{threeparttable}

\end{adjustwidth}


Según el algoritmo EMSR-b se comparan las clases en orden ascendente de precio. Se aplica la regla de Littlewood entre una clase y el resto cuál sea de mayor tarifa.
i.e. la clase E ($i = 5$) se compara con el conjunto de las clases A, B, C y D; la clase C ($i = 3$) se compara con A y B etc.\\


Entonces se obtiene la tarifa media ponderada de un conjunto de clases como:

\begin{align}
\text{Tarifa media ponderada}_i \equiv \text{tmp}_i = \dfrac{\sum_{k=1}^{i} \text{tarifa}_k \cdot \mu_k}{\sum_{k=1}^{i} \mu_k}
\end{align}

Para el conjunto de demanda se agregan las demandas:

\begin{align}
\overline{\mu}_i = \sum_{k=1}^{i} \mu_k
\end{align}

Para la desviación conjunta se toma la raíz cuadrada del sumatorio de las varianzas:

\begin{align}
\overline{\sigma}_i = \sqrt{\sum_{k=1}^i \sigma_k^2}
\end{align}


Dados estos parámetros conjuntos, se procede a aplicar la regla de Littlewood entre cada clase y el conjunto restante.

\begin{align}
\dfrac{\text{tarifa}_{i + 1}}{\text{tmp}_i} =  \underset{(\overline{\mu}_i, \, \overline{\sigma}_i)}{P(x > \theta_i)}
\end{align}


\begin{figure}
\centering
\begin{tikzpicture}
\begin{axis} [
title=Complementario,
axis lines = middle,
legend style={nodes={scale=0.5, transform shape}},
xlabel={$t$},
ylabel=$1 - ${$\underset{(\mu_i,\, \sigma_i)}{ P(t)}$},
ymax=1.1,
%ymin=5,
xmax=3,
xmin=-3,
%domain=3:20,
%samples=200,
width=12cm,height=4cm,
legend pos=north east,
]


\addplot [red, thick] table {data/data5.dat};
\addlegendentry{$(0,1)$}
\end{axis}


\end{tikzpicture}
\caption{Complementario de CDF}\label{fig:comcdf}
\end{figure}

Ha de encontrarse un valor $\theta_i$ para cada conjunto de clase, que representa el número de asientos protegidos, para el cuál la probabilidad de que tantos sean ocupados sea igual a la razón de tarifas.

\begin{align}
g(\theta) = \dfrac{1}{2} - \int_{\mu_i}^{\theta} \underset{(\overline{\mu}_i, \, \overline{\sigma}_i)}{f(x)}\dif x - \dfrac{\text{tarifa}_{i + 1}}{\text{tmp}_i}  = 0 \label{eq:gtheta}
\end{align}

Se puede resolver por linealización, i.e. Newton.\\




\begin{figure}
\centering
\begin{tikzpicture}
\begin{axis} [
title=Linealización,
axis lines = center,
legend style={nodes={scale=0.5, transform shape}},
xlabel={$\theta$},
ylabel=$1 - ${$\underset{(\overline{\mu}_i, \, \overline{\sigma}_i)}{ P(\theta)}$},
ymax=1.1,
%ymin=5,
xmax=3,
xmin=-3,
ticks=none,
%domain=3:20,
%samples=200,
width=12cm,height=6cm,
legend pos=north east,
]


\addplot [red, thick] table {data/data5.dat};
\addlegendentry{$(\overline{\mu}_i, \, \overline{\sigma}_i)$}

\addplot [blue, thick] {0.2};
\addlegendentry{$y=\text{tarifa}_{i+1}/\text{tmp}_i$}
\addplot [] {-0.399*(x-0)+0.5};
\addlegendentry{$\left(1-P(\theta)\right)\mathcal{O}(\theta^1)$}
\end{axis}

\end{tikzpicture}
\caption{Resolución de \eqref{eq:gtheta}}\label{fig:gtheta}
\end{figure}




Se resuelve para todos los valores EMSR y completa la tabla \ref{tab:emsr}.






\subsection{Valores preliminares}


\begin{adjustwidth}{-2.5 cm}{-2.5 cm}\centering
\begin{threeparttable}[!htb]

%\small
\begin{tabular}{lrcccccr}\toprule
Clase$_i$ & Tarifa$_i$ & $\mu_i$ &$\sigma_i$ & $\text{tmp}_i$ & $\overline{\mu}_i$ & $\overline{\sigma}_i$ & Protección\\\midrule
A$_1$ & 180 & \hl{ 14 } &\hl{ 9 } & 180 & 14 & 9 & 9 \\
B$_2$ & 130 & 87 & 8 & 137 & 101 & 12 & 94 \\
C$_3$ & 100 & 89 & 9 & 120 & 190 & 15 & 183 \\
D$_4$ & 80 & \hl{ 14 } &\hl{ 9 } & 117 & 204 & 18 & 211 \\
E$_5$ & 40 & 60 &9 & \hlc[cyan]{ (99) }   & \hlc[cyan]{ (264) } & \hlc[cyan]{ (20) } & \hlc[cyan]{ (289)* }\\
\bottomrule
\end{tabular}

\caption{Valores EMSR-b preliminares}\label{tab:emsrpre}
\end{threeparttable}

\end{adjustwidth}






\begin{figure}
\centering

\makebox[\textwidth]{
\begin{tikzpicture}
\begin{axis} [
% title=Tabla,
% axis lines = left,
axis y line*=left,
axis x line*=bottom,
legend style={nodes={scale=0.5, transform shape}},
xlabel={$\theta$},
ylabel=$f(\theta)_i$,
ymax=0.048,
ymin=0,
xmax=350,
xmin=0,
% ticks=none,
xtick={9,94,183,211,289},
    xticklabels={A, B, C, D, E*},
%domain=3:20,
%samples=200,
width=18cm,height=8cm,
legend pos=north east,
]


\addplot[blue, thick] table{data/data6.dat};
\addplot[cyan, thick] table{data/data7.dat};
\addplot[magenta, thick] table{data/data8.dat};
\addplot[teal, thick] table{data/data9.dat};
\addplot[red, thick] table{data/data10.dat};

\addplot[dashed,thin, red] coordinates {(9,0) (9,1)};
\addplot[dashed,thin, red] coordinates {(94,0) (94,1)};
\addplot[dashed,thin, red] coordinates {(183,0) (183,1)};
\addplot[dashed,thin, red] coordinates {(211,0) (211,1)};
\addplot[dashed,thin, red] coordinates {(289,0) (289,1)};


\end{axis}


\begin{axis} [
% axis lines = left,
axis y line*=right,
axis x line*=top,
legend style={nodes={scale=0.5, transform shape}},
% xlabel={$\theta$},
ylabel=tarifa$_i$,
% ymax=0.048,
ymin=0,
xmax=350,
xmin=0,
% ticks=none,
xtick={},
    xticklabels={},
% domain=0:400,
%samples=200,
width=18cm,height=8cm,
legend pos=north east,
grid=none,
]

\addplot [dashed, blue, thin, domain=9:400] {180};
\addplot [dashed, blue, thin, domain=94:400]  {130};
\addplot [dashed, blue, thin, domain=183:400]  {100};
\addplot [dashed, blue, thin, domain=211:400]  {80};
\addplot [dashed, blue, thin, domain=287:400]  {40};



\end{axis}


\end{tikzpicture}
  }


\caption{Distribuciones}\label{fig:tablaf}
\end{figure}



\begin{figure}
\centering

\makebox[\textwidth]{
\begin{tikzpicture}
\begin{axis} [
% title=Tabla,
axis lines = middle,
axis y line*=left,
legend style={nodes={scale=0.5, transform shape}},
xlabel={$\theta$},
ylabel=$g(\theta)_i$,
ymax=1,
%ymin=5,
xmax=350,
xmin=-14,
% ticks=none,
xtick={9,94,183,211,289},
    xticklabels={A, B, C, D, E*},
%domain=3:20,
%samples=200,
width=18cm,height=8cm,
legend pos=north east,
]


\addplot[blue, thick] table{data/data11.dat};
\addplot[cyan, thick] table{data/data12.dat};
\addplot[magenta, thick] table{data/data13.dat};
\addplot[teal, thick] table{data/data14.dat};
\addplot[red, thick] table{data/data15.dat};


\end{axis}
\end{tikzpicture}
  }


\caption{Distribuciones}\label{fig:tablag}
\end{figure}



\section{Vagones}

Han de determinarse el número óptimo de vagones a llevar. Sea entre un mínimo de 2 vagones, a un máximo de 4 vagones, configurados en la clase turista, con 80 plazas cada vagón.\\


Las tasas relativas al número de vagones y al número de pasajeros se recojen en la tabla \ref{tab:vag}.




\begin{table}[!htp]\centering

%\small
\begin{tabular}{lc}\toprule
Concepto & Tasa [\euro] \\\midrule
Tasa por pasajero & $1.5$\\
Coste por vagón & $500$\\
\bottomrule
\end{tabular}

\caption{Conceptos relativos a vagones y pasajeros}\label{tab:vag}
\end{table}


Sea la manera en la que se llenen los vagones, desde la clase más alta a la más baja, tal que no afecta a la demanda esperada ni a los niveles de protección.\\

Se define una función de coste y beneficio, $\gamma(\theta)$, la cuál recoge el beneficio neto en función del número de pasajeros $\theta$ y conversamente, del número de vagones aplicando las tasas asociadas.\\


Para cada valor de $\theta$ se atribuye un incremento de beneficio relativo al tipo de intervalo en el que se encuentra. Véase la figura \ref{fig:ingresos} y la tabla \ref{tab:emsrpre}. i.e. si $0 < \theta < A$ el beneficio a sumar por 1 pasajero extra corresponde a la tarifa A (180), menos la tasa de pasajeros. En total, para cada vagón se añade su tasa.


\begin{align}
\gamma(\theta) = \sum_{k=1}^\theta \underset{\left\{\min(i) \, \mid \, k \, < \, \theta_i^*\right \}}{\text{Tarifa}_i} - \left(\left\lceil\dfrac{\theta}{n_{\text{plazas}}}\right\rceil  \frac{\text{Coste}}{\text{vagón}} + \theta\frac{\text{Tasa}}{\text{pasajero}} \right) \label{eq:delta}
\end{align}

\begin{align}
\delta(\theta) = \sum_{k = 1}^\theta \underset{\left\{\min(i) \, \mid \, k \, < \, \theta_i^*\right \}}{\left[\left(\text{Tarifa}_i -\frac{\text{Tasa}}{\text{pasajero}}\right)\left(1 - \underset{(\overline{\mu}_i, \, \sigma_i)}{P(\theta)}\right)\right]} - \left\lceil\dfrac{\theta}{n_{\text{plazas}}}\right\rceil \frac{\text{Coste}}{\text{vagón}} \label{eq:deltastar}
\end{align}



\begin{figure}
\centering

\makebox[\textwidth]{
\begin{tikzpicture}
\begin{axis} [
% title=Tabla,
% axis lines = middle,
% axis y line*=left,
legend style={nodes={scale=0.5, transform shape}},
xlabel={$\theta$},
ylabel=$\gamma(\theta)$,
% ymax=0.048,
%ymin=5,
% xmax=350,
% xmin=0,
% ticks=none,
xtick={9,94,183,211,289, 80, 160, 240,320},
    xticklabels={A, B, C, D, E*, {\tiny 1er}, {\tiny 2o}, {\tiny 3er}, {\tiny 4o vg. completo}},
%domain=3:20,
%samples=200,
width=18cm,height=8cm,
legend pos=north west,
]




\addplot [blue, ultra thick] table{data/vagones0.dat};
\addlegendentry{$(1.5, 500)$}
\addplot [teal, thick] table{data/vagones1.dat};
\addlegendentry{$(1.5, 2000)$}
\addplot [olive, thick] table{data/vagones2.dat};
\addlegendentry{$(55, 500)$}
\addplot [purple, thick] table{data/vagones3.dat};
\addlegendentry{$(1.5, 5000)$}
\addplot [violet, thick] table{data/vagones4.dat};
\addlegendentry{$(0, 0)$}


\addplot[dashed,thin, red] coordinates {(9,0) (9,25000)};
\addplot[dashed,thin, red] coordinates {(94,0) (94,25000)};
\addplot[dashed,thin, red] coordinates {(183,0) (183,25000)};
\addplot[dashed,thin, red] coordinates {(211,0) (211,25000)};
\addplot[dashed,thin, red] coordinates {(289,0) (289,25000)};
\addplot[densely dotted, darkgray] coordinates {(80,0) (80,25000)};
\addplot[densely dotted, darkgray] coordinates {(160,0) (160,25000)};
\addplot[densely dotted, darkgray] coordinates {(240,0) (240,25000)};
\addplot[densely dotted, darkgray] coordinates {(320,0) (320,25000)};

\end{axis}



\end{tikzpicture}
  }


\caption{Distribución de ingresos en función del número de pasajeros y para distintas tasas}\label{fig:ingresos}
\end{figure}





\begin{figure}
\centering

\makebox[\textwidth]{
\begin{tikzpicture}
\begin{axis} [
% title=Tabla,
% axis lines = left,
axis y line*=left,
axis x line*=bottom,
legend style={nodes={scale=0.5, transform shape}},
xlabel={$\theta$},
ylabel=$\gamma(\theta)^{**}$,
% ymax=0.048,
% ymin=12000,
% xmax=350,
% xmin=0,
% ticks=none,
xtick={9,94,183,211,289, 80, 160, 240,320},
    xticklabels={A, B, C, D, E*, {\tiny 1er}, {\tiny 2o}, {\tiny 3er}, {\tiny 4o vg. completo}},
%domain=3:20,
%samples=200,
width=18cm,height=8cm,
legend pos=south west,
]

\addplot [blue, ultra thick, dashdotted] table{data/vagones0.dat};

\addplot [violet, ultra thick] table{data/vprob2.dat};


\addplot [RoyalBlue, ultra thick] table{data/vprob4.dat};

\addplot [magenta, densely dashdotted, thick] table{data/vprob5.dat};


% \addlegendentry{$(1.5, 500)$}

% \addlegendentry{$(1.5, 500)$}




\addplot[dashed, ultra thin, red] coordinates {(9,0) (9,25000)};
\addplot[dashed, ultra thin, red] coordinates {(94,0) (94,25000)};
\addplot[dashed, ultra thin, red] coordinates {(183,0) (183,25000)};
\addplot[dashed, ultra thin, red] coordinates {(211,0) (211,25000)};
\addplot[dashed, ultra thin, red] coordinates {(289,0) (289,25000)};
\addplot[densely dotted, darkgray] coordinates {(80,0) (80,25000)};
\addplot[densely dotted, darkgray] coordinates {(160,0) (160,25000)};
\addplot[densely dotted, darkgray] coordinates {(240,0) (240,25000)};
\addplot[densely dotted, darkgray] coordinates {(320,0) (320,25000)};

\end{axis}


\begin{axis} [
% axis lines = left,
axis y line*=right,
axis x line*=top,
legend style={nodes={scale=0.5, transform shape}},
% xlabel={$\theta$},
ylabel=$1 - P(\theta)$,
% ymax=1.19,
% ymin=0,
% xmax=350,
% xmin=0,
% ticks=none,
xtick={},
    xticklabels={},
% domain=0:400,
%samples=200,
width=18cm,height=8cm,
legend pos=north east,
grid=none,
]
\addplot [red, ultra thick, domain=0:200] {1};
\addplot [red, ultra thick] table{data/vprob0.dat};
% \addplot [ForestGreen, densely dashdotted, thick] table{data/vprob1.dat};

\addplot [yellow, densely dashdotted, ultra thick] table{data/vprob3.dat};

\addplot [ForestGreen, densely dashdotted, ultra thick] table{data/vprob6.dat};



\end{axis}


\end{tikzpicture}
  }


\caption{$\delta(\theta)$, según distintas funciones de probabilidad de demanda}\label{fig:distvag}
\end{figure}







\begin{figure}
\centering

\makebox[\textwidth]{
\begin{tikzpicture}
\begin{axis} [
% title=Tabla,
% axis lines = middle,
axis y line*=left,
axis x line*=bottom,
legend style={nodes={scale=0.5, transform shape}},
xlabel={$\theta$},
ylabel=$\delta(\theta)$,
% ymax=0.048,
%ymin=5,
% xmax=350,
% xmin=0,
% ticks=none,
xtick={9,94,183,211,289, 80, 160, 240,320},
    xticklabels={A, B, C, D, E*, {\tiny 1er}, {\tiny 2o}, {\tiny 3er}, {\tiny 4o vg. completo}},
%domain=3:20,
%samples=200,
width=18cm,height=8cm,
legend pos=south west,
]




\addplot [blue, ultra thick] table{data/vprob5.dat};
\addlegendentry{$(1.5, 500)$}
\addplot [teal, thick] table{data/delta1.dat};
\addlegendentry{$(1.5, 2000)$}
\addplot [olive, thick] table{data/delta2.dat};
\addlegendentry{$(55, 500)$}
\addplot [purple, thick] table{data/delta3.dat};
\addlegendentry{$(1.5, 5000)$}
\addplot [violet, thick] table{data/delta4.dat};
\addlegendentry{$(0, 0)$}


\addplot[dashed,thin, red] coordinates {(9,0) (9,25000)};
\addplot[dashed,thin, red] coordinates {(94,0) (94,25000)};
\addplot[dashed,thin, red] coordinates {(183,0) (183,25000)};
\addplot[dashed,thin, red] coordinates {(211,0) (211,25000)};
\addplot[dashed,thin, red] coordinates {(289,0) (289,25000)};
\addplot[densely dotted, darkgray] coordinates {(80,0) (80,25000)};
\addplot[densely dotted, darkgray] coordinates {(160,0) (160,25000)};
\addplot[densely dotted, darkgray] coordinates {(240,0) (240,25000)};
\addplot[densely dotted, darkgray] coordinates {(320,0) (320,25000)};

\end{axis}

\begin{axis} [
% axis lines = left,
axis y line*=right,
axis x line*=top,
legend style={nodes={scale=0.5, transform shape}},
% xlabel={$\theta$},
ylabel=$1 - P(\theta)$,
% ymax=1.19,
% ymin=0,
% xmax=350,
% xmin=0,
% ticks=none,
xtick={},
    xticklabels={},
% domain=0:400,
%samples=200,
width=18cm,height=8cm,
legend pos=north east,
grid=none,
]
\addplot [red, ultra thick, domain=0:200] {1};
\addplot [red, ultra thick] table{data/vprob0.dat};
% \addplot [ForestGreen, densely dashdotted, thick] table{data/vprob1.dat};

% \addplot [yellow, densely dashdotted, ultra thick] table{data/vprob3.dat};

\addplot [ForestGreen, densely dashdotted, ultra thick] table{data/vprob6.dat};



\end{axis}



\end{tikzpicture}
  }


\caption{$\delta(\theta)$ según distintas tasas}\label{fig:deltas}
\end{figure}



Similar a la distribución de ingresos según \eqref{eq:delta}, se pueden designar variaciones, como $\delta(\theta)$ \eqref{eq:deltastar}. Varias implementaciones se pueden ver en la figura \nolinebreak \ref{fig:distvag}.\\

La función $\delta(\theta)$ es una medida del ingreso esperado ponderado según la probabilidad para cada asiento. La probabilidad, las tasas, y las tarifas están delimitadas por los niveles de protección y corresponden individualmente a cada clase determinada por estos niveles.\\

Entonces el algoritmo aquí propuesto para determinar el número óptimo de vagones consiste en encontrar un valor máximo para la función $\delta(\theta)$, tal que la probabilidad para que $\theta^*$ asientos sean ocupados o más, sea superior a $0.5$. Véase en la figura \ref{fig:deltas}.


\begin{align}
N_{\text{vagones}} = \left\lceil\dfrac{\theta^*}{n_{\text{plazas}}}\right\rceil \, \mid \, \theta^* \, \mid \, \delta(\theta^*) > \delta(\theta) \,\, \& \,\, 1 - \underset{(\overline{\mu}_n , \, \overline{\sigma}_n)}{P(\theta^*)} > \dfrac{1}{2}
\end{align}



\end{document}
