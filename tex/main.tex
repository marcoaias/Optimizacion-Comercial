\documentclass[12pt]{article}


\usepackage{booktabs, multirow} % for borders and merged ranges
\usepackage{soul}% for underlines
\usepackage[table]{xcolor} % for cell colors
\usepackage{changepage,threeparttable}


\usepackage[official]{eurosym}


\usepackage{color}
% \usepackage[usenames,dvipsnames]{xcolor}

\newcommand{\hlc}[2][yellow]{ {\sethlcolor{#1} \hl{#2}} }

\usepackage{commath}

\usepackage[utf8]{inputenc}
\newtheorem{regla}{Regla}

\usepackage[spanish,es-tabla]{babel}

\usepackage{amsmath, amssymb, amsfonts}

\usepackage{tikz, pgfplots}
\pgfplotsset{compat=1.5}


\usepackage{hyperref}


% \newcommand{\hlc}[2][yellow]{ {\sethlcolor{#1} \hl{#2}} }





\begin{document}


\section{Productos}

%Please add the following packages if necessary:
%\usepackage{booktabs, multirow} % for borders and merged ranges
%\usepackage{soul}% for underlines
%\usepackage[table]{xcolor} % for cell colors
%\usepackage{changepage,threeparttable} % for wide tables
%If the table is too wide, replace \begin{table}[!htp]...\end{table} with
%\begin{adjustwidth}{-2.5 cm}{-2.5 cm}\centering\begin{threeparttable}[!htb]...\end{threeparttable}\end{adjustwidth}

\begin{adjustwidth}{-2.5 cm}{-2.5 cm}\centering
\begin{threeparttable}[!htb]

\small
\begin{tabular}{lccccc}\toprule
Clase de reserva &Tarifa [\euro] &Cambios permitidos &Sala Vip &Fast Track &Elección asiento \\\midrule
A &180 &Sí &Sí &Sí &Sí \\
B &130 &Solo 1 (penalización 25 \euro) &No &Sí &Sí \\
C &100 &Solo 1 (penalización 60 \euro) &No &No &Sí \\
D &80 &No &No &No &No \\
E &40 &No &No &No &No \\
\bottomrule
\end{tabular}
\caption{Servicios y precio de cada clase.}\label{tab:clases}
\end{threeparttable}

\end{adjustwidth}





\begin{table}[!htp]\centering

%\small
\begin{tabular}{lrrr}\toprule
Clase de reserva &$\mu$ &$\sigma$ \\\midrule
A &\hl{ ? } &\hl{ ? } \\
B &87 &8 \\
C &89 &9 \\
D &\hl{ ? } &\hl{ ? } \\
E &60 &9 \\
\bottomrule
\end{tabular}

\caption{Proyección de demanda para cada clase. (Trayecto MAD - BIO)}\label{tab:demanda}
\end{table}





\begin{align}
\sigma_1 &= e^{\mu / 7} + 2\label{eq:expbio}\\[8px]
\sigma_2 &= \dfrac{1}{10}\mu^{3} -20\mu + 20\label{eq:cubbio}
\end{align}

\vspace{10pt}

La tabla \ref{tab:demanda} y las ecuaciones \eqref{eq:expbio} y \eqref{eq:cubbio} corresponden al trayecto MAD - BIO; las ecuaciones aplican a las clases A y D.



\begin{figure}
\centering
\begin{tikzpicture}
\begin{axis} [
title=Curvas de demanda,
xlabel={$\mu$},
ylabel={$\sigma$},
ymax=20,
ymin=5,
xmax=20,
xmin=5,
%domain=5:35,
samples=200,
width=12cm, height=4cm,
legend pos=north west,
]






\addplot [red, thick,domain=5:35, restrict y to domain=1:30] {exp(x/7)+2};
\addlegendentry{$e^{\mu / 7} + 2$}

\addplot [blue, thick,domain=5:35, restrict y to domain=-90:120] {0.1*x^3 - 20*x + 20};
\addlegendentry{$0.1\mu^{3} -20\mu + 20$}



\end{axis}


\end{tikzpicture}
\caption{Representación de las ecuaciones \eqref{eq:expbio} y \eqref{eq:cubbio}}\label{fig:curvas}
\end{figure}


Sea el espacio de la solución:
$$\left\{5 \leq \mu \leq 35\right\}, \left\{1 \leq \sigma \leq 30\right\}$$



\begin{figure}
\centering
\begin{tikzpicture}
\begin{axis} [
title=Solución,
axis lines = middle,
xlabel={$\mu$},
ylabel={$\sigma_1 - \sigma_2$},
ymax=100,
ymin=-100,
%xmax=20,
%xmin=5,
domain=3:20,
samples=200,
width=12cm, height=4cm,
]

\addplot [cyan, thick] {exp(x/7)+2 - 0.1*x^3 + 20*x - 20};


\end{axis}


\end{tikzpicture}
\caption{Ecuación \eqref{eq:expbio} - \eqref{eq:cubbio}}\label{fig:sol}
\end{figure}




Según las figuras \ref{fig:curvas} y \ref{fig:sol}, la solución se encuentra en un entorno alrededor de $\mu \approx 14$.









\section{Probabilidad EMSR}


Para el cálculo de los EMSR(s), se requiere conocer la probabilidad de demanda; para lo cual el modelo toma los datos de demanda proyectados en la tabla \ref{tab:demanda} y forma una distribución normal de probabilidades.\\


La distribución gaussiana:

\begin{align}
f(x) &= \dfrac{1}{\sigma\sqrt{2\pi}}\exp\left({-\,\dfrac{1}{2}\left(\dfrac{x-\mu}{\sigma}\right)^2}\right) \label{eq:fgauss} \\[8pt]
P(t) &= \int_{-\infty}^{t} f(x) \dif x \underset{t\to +\infty}{=} 1 \label{eq:intgauss}
\end{align}


La integral \eqref{eq:intgauss} no se puede resolver de forma analítica, sino por aproximación por el \emph{método del trapecio} y también por el método de Simpson.\\



\begin{figure}
\centering
\begin{tikzpicture}
\begin{axis} [
title = Campana de Gauss,
axis lines = middle,
xlabel={$x$},
ylabel={$f(x)$},
ymax=0.5,
%ymin=5,
xmax=3,
xmin=-3,
%domain=3:20,
samples=200,
width=12cm,height=4cm,
]

\addplot [magenta, thick] {exp(-0.5*x^2)/(2*3.141592)^0.5};


\end{axis}


\end{tikzpicture}
\caption{Ecuación $\underset{\mu = 0,\,\sigma = 1}{\text{\eqref{eq:fgauss}}}$}\label{fig:campana}
\end{figure}



Dado que la función \eqref{eq:fgauss} es simétrica, la integral $\underset{t = \mu}{\text{\eqref{eq:intgauss}}}$ es igual a $\frac{1}{2}$. Entonces no es necesario calcular entre $(-\infty, t)$, sino $\frac{1}{2} + \int_\mu^t f(x)\dif x$ lo que alivia muchos recursos computacionales.

\begin{align}
P(t) = \dfrac{1}{2} + \int_\mu^t \underset{(\mu, \, \sigma)}{f(x)}\dif x \label{eq:probgauss}
\end{align}


\begin{figure}
\centering
\begin{tikzpicture}
\begin{axis} [
title=Probabilidad acumulada,
axis lines = middle,
legend style={nodes={scale=0.5, transform shape}},
xlabel={$t$},
ylabel={$\underset{(\mu,\, \sigma)}{P(t)}$},
ymax=1.1,
%ymin=5,
xmax=3,
xmin=-3,
%domain=3:20,
%samples=200,
width=12cm,height=4cm,
legend pos=south east,
]

\addplot [cyan, thick] table {data1.dat};
\addlegendentry{$(0,1)$}
\addplot [magenta, thick] table {data2.dat};
\addlegendentry{$(0,0.5)$}
\addplot [blue, thick] table {data3.dat};
\addlegendentry{$(1,1)$}
\addplot [red, thick] table {data4.dat};
\addlegendentry{$(-2,0.3)$}
\end{axis}


\end{tikzpicture}
\caption{Probabilidad acumulada en función de $t$ y $(\mu , \, \sigma)$ \eqref{eq:probgauss}}\label{fig:cdf}
\end{figure}



\section{Algoritmo EMSR-b}

Consiste en proteger suficientes \emph{asientos} de respectivas clases con tal de maximizar el ingreso total.\\

Se basa en la \emph{Regla de Littlewood}:

\begin{regla}
La razón entre las tarifas de respectivas clases ha de ser igual a la probabilidad de ser ocupado el asiento de la clase de la tarifa superior
\end{regla}


Ha de encontrarse el número de asientos protegidos ($t$) que satisfaga la probabilidad ($P(t)$) \eqref{eq:probgauss} de ser ocupados requerida por la razón de las respectivas tarifas.\\

Ampliamos la tabla \ref{tab:demanda} para calcular los parámetros relativos a los EMSR(s):\\





\begin{adjustwidth}{-2.5 cm}{-2.5 cm}\centering
\begin{threeparttable}[!htb]

%\small
\begin{tabular}{lrcccccr}\toprule
Clase$_i$ & Tarifa$_i$ & $\mu_i$ &$\sigma_i$ & $\text{tmp}_i$ & $\overline{\mu}_i$ & $\overline{\sigma}_i$ & Protección\\\midrule
A$_1$ & 180 & \hl{ ? } &\hl{ ? } & & & & \\
B$_2$ & 130 & 87 &8 & & & & \\
C$_3$ & 100 & 89 &9 & & & & \\
D$_4$ & 80 & \hl{ ? } &\hl{ ? } & & & & \\
E$_5$ & 40 & 60 &9 & - & - & - & -\\
\bottomrule
\end{tabular}

\caption{Valores EMSR-b}\label{tab:emsr}
\end{threeparttable}

\end{adjustwidth}


Según el algoritmo EMSR-b se comparan las clases en orden ascendente de precio. Se aplica la regla de Littlewood entre una clase y el resto cuál sea de mayor tarifa.
i.e. la clase E ($i = 5$) se compara con el conjunto de las clases A, B, C y D; la clase C ($i = 3$) se compara con A y B etc.\\


Entonces se obtiene la tarifa media ponderada de un conjunto de clases como:

\begin{align}
\text{Tarifa media ponderada}_i \equiv \text{tmp}_i = \dfrac{\sum_{k=1}^{i} \text{tarifa}_k \cdot \mu_k}{\sum_{k=1}^{i} \mu_k}
\end{align}

Para el conjunto de demanda se agregan las demandas:

\begin{align}
\overline{\mu}_i = \sum_{k=1}^{i} \mu_k
\end{align}

Para la desviación conjunta se toma la raíz cuadrada del sumatorio de las varianzas:

\begin{align}
\overline{\sigma}_i = \sqrt{\sum_{k=1}^i \sigma_k^2}
\end{align}


Dados estos parámetros conjuntos, se procede a aplicar la regla de Littlewood entre cada clase y el conjunto restante.

\begin{align}
\dfrac{\text{tarifa}_{i + 1}}{\text{tmp}_i} =  \underset{(\overline{\mu}_i, \, \overline{\sigma}_i)}{P(x > \theta_i)}
\end{align}


\begin{figure}
\centering
\begin{tikzpicture}
\begin{axis} [
title=Complementario,
axis lines = middle,
legend style={nodes={scale=0.5, transform shape}},
xlabel={$t$},
ylabel=$1 - ${$\underset{(\mu_i,\, \sigma_i)}{ P(t)}$},
ymax=1.1,
%ymin=5,
xmax=3,
xmin=-3,
%domain=3:20,
%samples=200,
width=12cm,height=4cm,
legend pos=north east,
]


\addplot [red, thick] table {data5.dat};
\addlegendentry{$(0,1)$}
\end{axis}


\end{tikzpicture}
\caption{Complementario de CDF}\label{fig:comcdf}
\end{figure}

Ha de encontrarse un valor $\theta_i$ para cada conjunto de clase, que representa el número de asientos protegidos, para el cuál la probabilidad de que tantos sean ocupados sea igual a la razón de tarifas.

\begin{align}
g(\theta) = \int_{\mu_i}^{\theta_i} \underset{(\overline{\mu}_i, \, \overline{\sigma}_i)}{f(x)}\dif x + \dfrac{\text{tarifa}_{i + 1}}{\text{tmp}_i} - \dfrac{1}{2} = 0 \label{eq:gtheta}
\end{align}

Se puede resolver por linealización, i.e. Newton.\\




\begin{figure}
\centering
\begin{tikzpicture}
\begin{axis} [
title=Linealización,
axis lines = center,
legend style={nodes={scale=0.5, transform shape}},
xlabel={$\theta$},
ylabel=$1 - ${$\underset{(\overline{\mu}_i, \, \overline{\sigma}_i)}{ P(\theta)}$},
ymax=1.1,
%ymin=5,
xmax=3,
xmin=-3,
ticks=none,
%domain=3:20,
%samples=200,
width=12cm,height=6cm,
legend pos=north east,
]


\addplot [red, thick] table {data5.dat};
\addlegendentry{$(\overline{\mu}_i, \, \overline{\sigma}_i)$}

\addplot [blue, thick] {0.2};
\addlegendentry{$y=\text{tarifa}_{i+1}/\text{tmp}_i$}
\addplot [] {-0.399*(x-0)+0.5};
\addlegendentry{$\left(1-P(\theta)\right)\mathcal{O}(\theta^1)$}
\end{axis}


\end{tikzpicture}
\caption{Resolución de \eqref{eq:gtheta}}\label{fig:gtheta}
\end{figure}




Se resuelve para todos los valores EMSR y completa la tabla \ref{tab:emsr}.






\subsection{Valores preliminares}


\begin{adjustwidth}{-2.5 cm}{-2.5 cm}\centering
\begin{threeparttable}[!htb]

%\small
\begin{tabular}{lrcccccr}\toprule
Clase$_i$ & Tarifa$_i$ & $\mu_i$ &$\sigma_i$ & $\text{tmp}_i$ & $\overline{\mu}_i$ & $\overline{\sigma}_i$ & Protección\\\midrule
A$_1$ & 180 & \hl{ 14 } &\hl{ 9 } & 180 & 14 & 9 & 9 \\
B$_2$ & 130 & 87 & 8 & 137 & 101 & 12 & 94 \\
C$_3$ & 100 & 89 & 9 & 120 & 190 & 15 & 183 \\
D$_4$ & 80 & \hl{ 14 } &\hl{ 9 } & 117 & 204 & 18 & 211 \\
E$_5$ & 40 & 60 &9 & \hlc[cyan]{ (99) }   & \hlc[cyan]{ (264) } & \hlc[cyan]{ (20) } & \hlc[cyan]{ - }\\
\bottomrule
\end{tabular}

\caption{Valores EMSR-b preliminares}\label{tab:emsrpre}
\end{threeparttable}

\end{adjustwidth}





\end{document}
