\documentclass[12pt]{article}


\usepackage{booktabs, multirow} % for borders and merged ranges
\usepackage{soul}% for underlines
\usepackage[table]{xcolor} % for cell colors
\usepackage{changepage,threeparttable}


\usepackage[official]{eurosym}


\usepackage{commath}

\usepackage[utf8]{inputenc}
\newtheorem{regla}{Regla}

\usepackage[spanish,es-tabla]{babel}

\usepackage{amsmath, amssymb, amsfonts}

\usepackage{tikz, pgfplots}
\pgfplotsset{compat=1.5}


\usepackage{hyperref}





\begin{document}


\section{Productos}

%Please add the following packages if necessary:
%\usepackage{booktabs, multirow} % for borders and merged ranges
%\usepackage{soul}% for underlines
%\usepackage[table]{xcolor} % for cell colors
%\usepackage{changepage,threeparttable} % for wide tables
%If the table is too wide, replace \begin{table}[!htp]...\end{table} with
%\begin{adjustwidth}{-2.5 cm}{-2.5 cm}\centering\begin{threeparttable}[!htb]...\end{threeparttable}\end{adjustwidth}

\begin{adjustwidth}{-2.5 cm}{-2.5 cm}\centering
\begin{threeparttable}[!htb]

\small
\begin{tabular}{lccccc}\toprule
Clase de reserva &Tarifa [\euro] &Cambios permitidos &Sala Vip &Fast Track &Elección asiento \\\midrule
A &180 &Sí &Sí &Sí &Sí \\
B &130 &Solo 1 (penalización 25 \euro) &No &Sí &Sí \\
C &100 &Solo 1 (penalización 60 \euro) &No &No &Sí \\
D &80 &No &No &No &No \\
E &40 &No &No &No &No \\
\bottomrule
\end{tabular}
\caption{Servicios y precio de cada clase.}\label{tab:clases}
\end{threeparttable}

\end{adjustwidth}





\begin{table}[!htp]\centering

%\small
\begin{tabular}{lrrr}\toprule
Clase de reserva &$\mu$ &$\sigma$ \\\midrule
A &\hl{ ? } &\hl{ ? } \\
B &87 &8 \\
C &89 &9 \\
D &\hl{ ? } &\hl{ ? } \\
E &60 &9 \\
\bottomrule
\end{tabular}

\caption{Proyección de demanda para cada clase. (Trayecto MAD - BIO)}\label{tab:demanda}
\end{table}





\begin{align}
\sigma &= e^{\mu / 7} + 2\label{eq:expbio}\\[8px]
\sigma &= \dfrac{1}{10}\mu^{3} -20\mu + 20\label{eq:cubbio}
\end{align}

\vspace{10pt}

La tabla \ref{tab:demanda} y las ecuaciones \eqref{eq:expbio} y \eqref{eq:cubbio} corresponden al trayecto MAD - BIO; las ecuaciones aplican a las clases A y D.



\begin{figure}
\centering
\begin{tikzpicture}
\begin{axis} [
title=Curvas de demanda,
xlabel={$\mu$},
ylabel={$\sigma$},
ymax=20,
ymin=5,
xmax=20,
xmin=5,
%domain=5:35,
samples=200,
width=12cm, height=4cm,
legend pos=north west,
]






\addplot [red, thick,domain=5:35, restrict y to domain=1:30] {exp(x/7)+2};
\addlegendentry{$e^{\mu / 7} + 2$}

\addplot [blue, thick,domain=5:35, restrict y to domain=-90:120] {0.1*x^3 - 20*x + 20};
\addlegendentry{$0.1\mu^{3} -20\mu + 20$}



\end{axis}


\end{tikzpicture}
\caption{Representación de las ecuaciones \eqref{eq:expbio} y \eqref{eq:cubbio}}\label{fig:curvas}
\end{figure}


Sea el espacio de la solución:
$$\left\{5 \leq \mu \leq 35\right\}, \left\{1 \leq \sigma \leq 30\right\}$$



\begin{figure}
\centering
\begin{tikzpicture}
\begin{axis} [
title=Solución,
axis lines = middle,
xlabel={$\mu$},
ylabel={$\sigma$},
ymax=100,
ymin=-100,
%xmax=20,
%xmin=5,
domain=3:20,
samples=200,
width=12cm, height=4cm,
]

\addplot [cyan, thick] {exp(x/7)+2 - 0.1*x^3 + 20*x - 20};


\end{axis}


\end{tikzpicture}
\caption{Ecuación \eqref{eq:expbio} - \eqref{eq:cubbio}}\label{fig:sol}
\end{figure}




Según las figuras \ref{fig:curvas} y \ref{fig:sol}, la solución se encuentra en un entorno alrededor de $\mu \approx 14$.









\section{Probabilidad EMSR}


Para el cálculo de los EMSR(s), se requiere conocer la probabilidad de demanda; para lo cual el modelo toma los datos de demanda proyectados en la tabla \ref{tab:demanda} y forma una distribución normal de probabilidades.\\


La distribución gaussiana:

\begin{align}
f(x) &= \dfrac{1}{\sigma\sqrt{2\pi}}\exp\left({-\,\dfrac{1}{2}\left(\dfrac{x-\mu}{\sigma}\right)^2}\right) \label{eq:fgauss} \\[8pt]
P(t) &= \int_{-\infty}^{t} f(x) \dif x \underset{t\to +\infty}{=} 1 \label{eq:intgauss}
\end{align}


La integral \eqref{eq:intgauss} no se puede resolver de forma analítica, sino por aproximación por el \emph{método del trapecio} y también por el método de Simpson.\\



\begin{figure}
\centering
\begin{tikzpicture}
\begin{axis} [
axis lines = middle,
xlabel={$x$},
ylabel={$f$},
ymax=0.5,
%ymin=5,
xmax=3,
xmin=-3,
%domain=3:20,
samples=200,
width=12cm,height=4cm,
]

\addplot [magenta, thick] {exp(-0.5*x^2)/(2*3.141592)^0.5};


\end{axis}


\end{tikzpicture}
\caption{Ecuación $\underset{\mu = 0,\,\sigma = 1}{\text{\eqref{eq:fgauss}}}$}\label{fig:campana}
\end{figure}



Dado que la función \eqref{eq:fgauss} es simétrica, la integral $\underset{t = \mu}{\text{\eqref{eq:intgauss}}}$ es igual a $\frac{1}{2}$. Entonces no es necesario calcular entre $(-\infty, t)$, sino $\frac{1}{2} + \int_\mu^t f(x)\dif x$ lo que alivia muchos recursos computacionales.

\begin{align}
P(t) = \dfrac{1}{2} + \int_\mu^t f(x)\dif x
\end{align}


\section{Algoritmo EMSR-b}

Consiste en proteger suficientes \emph{asientos} de respectivas clases con tal de maximizar el ingreso total.\\

Se basa en la \emph{Regla de Littlewood}:

\begin{regla}
La razón entre las tarifas de respectivas clases ha de ser igual a la probabilidad de ser ocupado el asiento de la clase de la tarifa superior
\end{regla}




\end{document}
